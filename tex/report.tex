\documentclass{article}
\usepackage{graphicx} % Para incluir imágenes
\usepackage{amsmath}  % Para matemáticas
\usepackage{pgfplots}
\usepackage{float}
\pgfplotsset{compat=1.16}
\title{Proyecto de Simulación: Masa y Resorte}
\author{Lucio Mansilla , Brenda Dichiara}
\date{\today}

\begin{document}

\maketitle

\section{Introducción}
En este proyecto, exploramos la dinámica de un sistema masa-resorte, con el objetivo de comprender cómo las condiciones iniciales y los parámetros del sistema afectan su comportamiento. Utilizamos el método de Euler para resolver numéricamente las ecuaciones diferenciales que describen el sistema. En este trabajo se observo cómo la masa, la constante del resorte, la resistencia al rozamiento y una fuerza externa aplicada al modelo, afectan a la posición, velocidad oscilaciones, discipación de energía y puntos de equlibrio a lo largo del tiempo.

\section{Modelo y Método de Solución}
El modelo físico que utilizamos es el de un resorte con una masa $m$ sujeta a él, una constante de resistencia del resorte $k$, una resistencia al rozamiento $b$, y una fuerza externa $F$.

\vspace{0.1cm}
\begin{center}
\begin{tikzpicture}[scale=1.5]

    % Masa
    \draw[fill=white,opacity=0.7] (2,-0.5) rectangle ++(1,1) node[midway] {$m$};
    % Resorte
    \draw[decoration={coil},decorate] (0,0) -- (2,0) node[midway, above=5mm] {$k$};
    % Pared
    \draw[very thick] (0,-1) -- (0,1);
    % Fuerza
    \draw[->,thick, red] (3,0) -- ++(1,0) node[midway,above] {$F(t)$};
    % Rozamiento
    \draw[->,thick, blue] (2,-0.5) -- ++(-0.5,0) node[midway,below] {$b$};
    % Coordenadas
    \draw[->] (0,-1) -- ++(5,0) node[below] {$x(t)$}; % posicion
    \draw[->] (0,-1) -- ++(0,2) node[left] {$y(t)$}; % velocidad
\end{tikzpicture}
\end{center}
Las ecuaciones diferenciales que modelan este sistema continuo son:
\begin{align*}
    \frac{dx}{dt} & = v \\\\ 
    \frac{dv}{dt} & = -\frac{k}{m}x - \frac{b}{m}v + \frac{F}{m}.
\end{align*}

\section{Validación del Modelo}
Para validar el modelo de simulación de masa-resorte, necesitamos comparar los resultados de la simulación con los resultados experimentales o con soluciones analíticas conocidas del sistema.

\subsection{Comparación con Soluciones Analíticas}
% La ecuación diferencial que describe un sistema masa-resorte sin amortiguamiento es una ecuación bien conocida cuya solución es una función sinusoidal o cosinusoidal. Si incluimos el amortiguamiento, la solución se convierte en una exponencial amortiguada multiplicada por una sinusoidal o cosinusoidal. Podemos comparar los resultados de nuestra simulación con estas soluciones analíticas para verificar que nuestro modelo esté funcionando correctamente. 


\section{Experimentos}
Realizamos una serie de experimentos variando la masa $m$, la constante del resorte $k$, la resistencia al rozamiento $b$, la fuerza externa $F$, y las condiciones iniciales de posición  , velocidad $x_0$ ,$v_0$ y el "paso" de tiempo $\Delta t$. A continuación, presentamos los resultados más interesantes de estos experimentos.

\subsection{Experimento 1: Variación de la masa}
En este experimento, nos centramos en el impacto en  la variación de la masa $m$ en la dinámica del sistema. Para asegurar un control riguroso sobre las variables de estudio, mantuvimos constantes el resto de los parámetros:

\begin{itemize}
\item Constante del resorte, $k = 1.0$
\item Resistencia al rozamiento, $b = 1.0$
\item Fuerza externa, $F = 1.0$
\item Posición inicial, $x_0 = 0.0$
\item Velocidad inicial, $v_0 = 0.0$
\end{itemize}

Luego, implementamos simulaciones con distintos valores de masa, en particular $m = 1.0$ y $m = 3.0$. El paso de tiempo seleccionado para la simulación fue $\Delta t= 0.001$.

Las simulaciones realizadas con 150 como límite de tiempo $t$ mostraron que a medida que la masa aumenta, el sistema tarda un mayor tiempo en alcanzar un estado de equilibrio, como se puede ver en la figura 1.

\begin{figure}[H]
\centering
%\includegraphics[width=0.8\textwidth]{images/mass.png}
\caption{Posición - Velocidad en función del tiempo para distintos valores de masa, $t = 100$.}
\end{figure}

Este fenómeno se debe a que la fuerza aplicada, es constante i.e $F = 1$, por lo se necesita más tiempo para generar suficiente impulso y mover la masa aumentada hacia el estado de equilibrio.

Mismo experimento pero aumentando el límite de tiempo 
$t$ también mostro que el modelo con mayor masa tiene más oscilaciones antes de alcanzar el estado de equilibrio, como se puede ver en la figura 2.

\begin{figure}[H]
    \centering
    %\includegraphics[width=0.8\textwidth]{images/mass.png}
    \caption{Posición - Velocidad en función del tiempo para distintos valores de masa, $t = 150$.}
\end{figure}
    

Se presentan los resultados observados de la figura 1 y 2 en la tabla 1.


\subsection{Experimento 2: Ausencia de fricción}
En este segundo experimento, nos propusimos examinar el comportamiento del sistema de masa-resorte en ausencia de fricción. Para ello, se estableció la resistencia al rozamiento $b$ en cero, manteniendo constantes los demás parámetros:

\begin{itemize}
\item Masa, $m = 1.0$
\item Constante del resorte, $k = 1.0$
\item Fuerza externa, $F = 1.0$
\item Posición inicial, $x_0 = 0.0$
\item Velocidad inicial, $v_0 = 0.0$
\end{itemize}

El paso de tiempo para la simulación fue nuevamente de $\Delta t= 0.001$.

La ausencia de fricción resultó en un sistema que nunca alcanza un estado de equilibrio, como se muestra en la Figura Y. Sin fricción para disipar la energía inyectada constantemente al sistema por la fuerza externa $F$, el sistema continúa oscilando sin cesar. Esto resalta el papel crucial de la fricción como un factor estabilizador en sistemas dinámicos con fuerza externa constante.

Además, al aumentar la masa a $m = 2.0$, se observó una disminución en la frecuencia de las oscilaciones, tal como se muestra en la Figura Z. Esto se debe a que una mayor masa requiere más fuerza para cambiar su estado de movimiento, lo que resulta en oscilaciones de menor frecuencia.



\subsection{Experimento 3: Ausencia de Fricción con Masa Variable}



\subsection{Experimento 4: Desplazamiento del estado de equilibrio}

\section{Conclusiones}
Conclusiones basadas en los resultados de los experimentos...


\end{document}
